\section{Performance Evaluation}
% \IEEEPARstart{T}{his} demo file is intended to serve as a ``starter file''
% for IEEE Computer Society journal papers produced under \LaTeX\ using
% IEEEtran.cls version 1.7 and later.
% % You must have at least 2 lines in the paragraph with the drop letter
% % (should never be an issue)
% I wish you the best of success.
% \hfill mds
% \hfill January 11, 2007

\subsection{Experimental Setup}
Specify the experimental setup (e.g., OS, Linux version,
kernel version, CPU spec, DRAM size, storage devices, etc.) 
and benchmark setup (e.g., database size, \# of concurrent threads, 
running time).

System Setup
\begin{center}
    \begin{tabular}{ | c | c | }
    \hline
        \textbf{Type} & \textbf{Specification} \\
    \hline
        OS & Ubuntu 18.04.6 LTS \\
    \hline
        CPU & \makecell{12th Gen Intel(R) Core(TM) \\ i7-12650H 
        (Total 2 cores)} \\
    \hline
        Memory & 4GB \\
    \hline
        Kernel & 5.4.0-84-generic \\
    \hline
        Data Device & Samsung PM991a SSD 1TB \\
    \hline
    \end{tabular}
\end{center}
%
% Benchmark Setup templates
Benchmark Setup
\begin{center}
    \begin{tabular}{ | c | c | }
    \hline
        \textbf{Type} & \textbf{Configuration} \\
    \hline
        DB Size & 2GB (20 warehouse) \\
    \hline
        Buffer Pool Size & 500MB (25\% of DB Size) \\
    \hline
        Benchmark Tool & tpcc-mysql \\
    \hline
        Runtime & 1200s \\
    \hline
        Connections & 8 \\
    \hline
    \end{tabular}
\end{center}

\subsection{Experimental Results}
Present experimental results and analyze the results yourself.